The 21st century brought several challenges to improve energy consumptions, increase the usage of renewable energy sources, increase the quality of service, reduce the operation costs, etc. All of these challenges are proposed to reduce the fossil energy dependence, reduce the energy consumption and minimize the environment damaging.

The European Commission 2020 targets are clear: reducing 20\% of greenhouse emissions, increase 20\% of energy from renewable sources and improve 20\% of energy efficiency. Smart Grids are being developed to be a technology that, by increasing the energy efficiency and increasing the usage of renewable energy sources, it is possible to reduce the greenhouse gas emissions and fulfill the 2020 targets.

An important part of the SG are the metering systems. Starting with smart (and non-smart) meters, the electrical energy measurement devices was presented, covering the background behind smart meters and finalizing with regulations and directives concerning these devices.
The Smart Metering Infrastructure is presented to have a higher view of integration in the Smart Grid. Followed is the Energy Management Systems and Energy Information Systems. A cost-analysis is necessary to evaluate the Smart Metering Systems integration in the traditional electrical grid. 

On the field of the communications, a three stage  state of the art covering approach was followed: requirements, followed by the presentation of the communication technologies and ending with the standards. The content of this paper has ended with the presentation of some opportunities and trends on research of smart metering, giving a focus to present the opportunities in the railway environments.
To conclude, the future of smart metering systems is bright with the incentives from the European Commission, specifically from Shift2Rail program.
