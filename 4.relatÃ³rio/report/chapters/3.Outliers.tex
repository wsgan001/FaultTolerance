%\lipsum[4-4]
In this chapter it is presented the state of the art regarding metering systems in the electrical energy field. These systems are divided in metering systems, Advanced Metering Infrastructure (AMI), Energy Management Systems (EMSs) and Energy Information Systems (EIS).

%%%%%%%%%%%%%%%%%%%%%
%%   Smart metering Systems
%%%%%%%%%%%%%%%%%%%%%
\section{Smart Meters}

\input{chapters/B11-SM}

%\newpage
\vspace{3em}
%%%%%%%%%%%%%%%%%%%%%
%%   Advanced metering infrastructure
%%%%%%%%%%%%%%%%%%%%%
\section{Advanced Metering Infrastructure}

\input{chapters/B12-AMI}



\vspace{3em}

%%%%%%%%%%%%%%%%%%%%%
%%   Energy management systems
%%%%%%%%%%%%%%%%%%%%%
\section{Energy Management Systems}

\input{chapters/B13-EMS}


\vspace{3em}

%%%%%%%%%%%%%%%%%%%%%
%%   Energy information systems
%%%%%%%%%%%%%%%%%%%%%
\section{Energy Information Systems}

\input{chapters/B14-EIS}

\vspace{3em}
%%%%%%%%%%%%%%%%%%%%%
%%   Cost-benefit analysis of Smart Metering
%%%%%%%%%%%%%%%%%%%%%
\section{Cost-benefit Analysis of Smart Metering Systems}

\input{chapters/B15-CB}


\vspace{3em}
%%%%%%%%%%%%%%%%%%%%%
%%   Cost-benefit analysis of Smart Metering
%%%%%%%%%%%%%%%%%%%%%
\section{Synthesis}

This chapter follow the structure proposed by \cite{Siano2014}. There are several approaches to cover this thematic, however the structure followed allows to have a better view of the main component of a smart system - the smart meter - before went to the tudy of the overall system.

Since there is an important background on smart meters, the electromechanical energy readers and later on automatic metering readers, it was followed a study on this background towards presenting the smart meter. Finally, a rough view on the regulations applied on energy meters are presented. 

The whole architecture - or infrastructure - was then presented, having in mind the proposal of \cite{Siano2014} on a good structure to present this thematic. In this section was discarded the presentation of the communication technologies that support this infrastructure, since these technologies will better presented in chapter 3.

However, this architecture is open and does not follow a strict regulation, specially i terms of definitions or keywords. This means that several approaches could be presented to cover this "infrastructure". Nevertheless, that does not mean that a bad coverage of this infrastructure could be present by other authors. Also, since this architecture has been emerging from the central infrastructure of the utility grid, where all the management is made a grid operator, several coverages of this topic may follow a not well defined definition of smart grid/smart metering system.

As it was presented in the beginning and following the ideas of several authors that cover this thematic, there is no exact definition of smart meters, smart grid, smart monitoring infrastructure, etc. It was made an assumption for smart metering systems, whee those systems may act as s system where each node has a bi-directional mean of communication. Therefore, a distinction is made between the "old" metering devices - where the utility grid/supervisor only have access to the metering data - and the new smart metering devices that has a bi-directional communication, receive information/commands from the utility grid/supervisor of the smart grid/etc. and, some kind of decision making could be done by those devices (as an example, promote energy efficiency by enabling/disabling devices that could operate in off-peak hours, such as heating devices, freezing devices, etc).

A brief effort was made to present the future directions of this study on railway systems. This complex environment could be considered as a possible smart grid and an advanced metering and monitoring infrastructure inherent to this environment. Therefore, a study on the Energy Management Systems for railways was made as a starting point to the future work.

On the following chapter, the communications will be covered to fulfill the gap left by the coverage of this metering infrastructure chapter.