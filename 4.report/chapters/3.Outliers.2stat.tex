\newpage

\section{Statistical based techniques}
\label{sec:statbased}



\cite{gen:chandola:2009} identifies the statistical techniques for anomaly detection based on the assumption that, in a stochastic model, the most common data instances occur in high probability regions and the anomalies occur in low probability regions.

To detect anomalies, parametric techniques are suggested since those techniques \textbf{assume} the knowledge of the underlying distribution and \textbf{estimate} the parameters from a given data set. Non-parametric techniques differs from parametric ones without the need of assuming the knowledge of the distribution.

\cite{cluster:andrade2016} lists some statistical parametric techniques: 

\begin{description}
	\setlength\itemsep{-0.5em}
	\item[Peirce's Criterion]  
	Statistical parametric method for outlier detection based on a normal distribution.
	\item[Chauvenet's Criterion]  
	This criterion is based on the assumption a given measurement may
	be rejected if the probability of obtaining the deviation for this average value is less than the inverse of twice the number of measurements.
\end{description}

%%%%%%%%%%%%%%%
%%%%%%%%%%%%%%%
\subsection{Parametric --- Gaussian based}
\cite{gen:zhang:2010} summarizes parametric techniques as an anomaly detection strategy based on the following steps:

\begin{itemize}
	\setlength\itemsep{-0.5em}
	\item The available knowledge is generated from a known distribution;
	\item The distribution parameters is then estimated from the give data.	
\end{itemize}

The usage of Gaussian  models allows the spatial correlation of the readings towards distinguishing between outlying sensors and event boundary. 

\subsection{Non-parametric --- Histogram based}
\cite{stat:sheng:2007} proposes a histogram-based method to reduce the communication cost on sensor networks. The main objective of this proposal is to collect hints (in a form of histogram) about the data distribution and, with the knowledge from these hints, unnecessary data is filtered and potential outliers are detected 


Complementary, \cite{stat:wang:2013} introduces clusters on incremental histogram scheme based on a divide and conquer strategy:

\begin{itemize}
	\setlength\itemsep{-0.5em}
	\item The wireless network is divided in clusters (based on adjacent nodes and data correlated strategy);
	\item The cluster head and cluster members updates the histogram incrementally and compares histograms in the form of Kullback-leibler divergence differentially (Kullback-leibler divergence is a convenient and robust
	method of measuring the difference between two data sets in a
	statistical sense.)
\end{itemize}

%%%%%%%%%%%%%%%
%%%%%%%%%%%%%%%
\subsection{Non-parametric --- Kernel function based}
\cite{gen:zhang:2010} synthesizes the concept of Kernel function non-parametric approaches as methods for estimating the probability distribution function for normal instances (and a new instance that occurs on a low probability area is declared an outlier). Later on in section \ref{sec:specbased} is presented the Kernel Principal Component Analysis (KPCA) used by \cite{stat:ghorbel:2014} for outlier detection in WSN's.

\cite{cluster:andrade2016} identify some kernel regression techniques: 

\begin{description}
	\setlength\itemsep{-0.5em}
	\item[Marzullo's Fault Tolerant Sensor Averaging (FTA) ] 
	Simple method for sensor fusion where the data assumed as anomalous is deleted.
	\item[Elmenreich's Confidence-Weighted Averaging (CWA) ] 
	The sensor's confidence are correlated with the sensor's variance.
	\item[CWA+FTA method ]
	This method combines both methods where the confidence-weighted average is calculated and two-thirds of the anomalous data is removed.
\end{description}
%%
%%
