\newpage


\section{Clustering based techniques}
\label{sec:clustbased}
%\lipsum[4-4]

\cite{gen:chandola:2009} synthesizes the clustering techniques in three categories based on three different assumptions: 

\begin{itemize}
	
	\setlength\itemsep{-0.5em}
	
	\item The normal data instances are part of a cluster in the data and the outliers does not fit any cluster;
	
	\item The normal data instances are present close to it's closest cluster centroid and the outliers lies far way from their closest cluster centroid;
	
	\item The normal data instances are part of large dense clusters and the outliers are part of small or sparse clusters.
	
\end{itemize}

\cite{clust:rajasegarar:2006} uses a technique to minimize the communication overhead by using clusters among the sensor readings. In a further step, it merges the clusters before the data is sent to other nodes.
