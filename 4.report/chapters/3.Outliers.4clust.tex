\newpage


\section{Clustering based techniques}
\label{sec:clustbased}
%\lipsum[4-4]

\cite{gen:chandola:2009} synthesizes the clustering techniques in three categories based on three different assumptions: 

\begin{itemize}
	
	\setlength\itemsep{-0.5em}
	
	\item The normal data instances are part of a cluster in the data and the outliers does not fit any cluster;
	
	\item The normal data instances are present close to its closest cluster centroid and the outliers lies far away from their closest cluster centroid;
	
	\item The normal data instances are part of large dense clusters and the outliers are part of small or sparse clusters.
	
\end{itemize}

\cite{clust:rajasegarar:2006} uses a technique to minimize the communication overhead by using clusters among the sensor readings. In a further step, it merges the clusters before the data is sent to other nodes.

\cite{cluster:andrade2016} presents a methodology to apply clustering and statistical techniques. The clusters are grouped according to the spatial position of the sensors and a k-means nearest-neighbor technique is used to provide a better understanding of the sensed environment. The proposed methodology follows a two-step procedure, starting with the usage of clustering information and followed by a statistical-based method. The statistical method is Elmenreich's Confidence-Weighted Averaging (CWA), where the sensor's confidence is correlated with the sensor's variance.

\cite{clust:cenedese2017} considers the network decomposition (i.e. the communication network topology) together with the data clustering measurements. They propose two algorithms: a centralized clustering algorithm (CCA) and a distributed clustering algorithm (DCA).