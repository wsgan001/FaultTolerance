%\lipsum[4-4]
In this chapter it is  made the study of the state of the art of outliers and it's relevance in railways.


\section{Definition of outlier detection}

Outlier detection is a computational task to detect and retrieve information from erroneous data values. The definition is usually close to anomaly detection or deviation detection. 


%%%%%%%%%%%%%%%%%%%%%
%%   Wireless sensor networks
%%%%%%%%%%%%%%%%%%%%%
\section{Outlier detection in WSNs}

Wireless sensor networks (WSNs) has been widely used in several applications in several domains such as industrial, scientific, medical and others. Those applications has been supported by the advances in wireless technologies as well as in the evolution of microcontroller technologies, with enhanced processing capabilities associated with reduced energy consumption.

\subsection{Motivation}

"In sensor networks, the majority of the energy is consumed in radio communication rather than computation" ... in the particular case of Sensoria sensors and Berkeley motes, the ratio of energy consumption between computation and communication modes is between 1000 and 10000 <rajasegarar2007>. Thus, an research opportunity is raised to reduce the communication usage of $\mu C$s by adding processing features towards the redution of energy consumption.

The motivation of detecting outliers in data acquired from WSNs has been extensivelly presented in the literature. The need for acquire data from harsh or "highly dynamic" environments as well as the need to validate and extract knowledge from the acquired data are one of the main points in the motivation to study the outlier detection in WSNs, \cite{gen:zhang:2010}. <zang2010> <chandola2009> <ghorbel2015> <martins2015>  


\subsection{Research areas}
Zhang et al. <zhang2010> identifies the outlier detection research areas in three domains: 

\begin{itemize}
	\item Intrusion detection: Situation caused by malicious attacks, where the detection techniques are query-driven techniques;
	
	\item Fault detection: Situation where the data suffer from noise and errors and where the detection techniques are data-driven ones;
	
	\item Event detection: Situation caused by the occurrence of one atomic or multiple events and where the majority of the research has been developed due to the complexity of detecting and extracting information on ?? upper layers ??
\end{itemize}

Based on the division of this three domains, the upcoming research is intended to be focused on the event detection techniques. ?? The railway environment requires closed subsystems that meets specific standards. Despite the intrusion detection should be considered, this must be took into consideration accordingly to the development and implementation of the wireless smart metering system for the railways. ?? The fault detection should and must be taken into consideration and the data outcome must be, preferably, a null value with a warning raised. ???

\subsection{Challenges}

The challenges of outlier detection in WSNs are related to the quality of the acquisition of the sensors, the fiability of the modules in terms of energy or environmental susceptibility, and the communication requirements and restrictions.

Zhang et al. <zhang2010> lists the challenges as the following:

\begin{itemize}
	\setlength\itemsep{-0.5em}
	
	\item Resource constraints;
	
	\item High communication costs;
	
	\item Distributed streaming data;
	
	\item Dynamic network topology, \\ frequent communication failures, \\ mobility and heterogeneity of nodes;
	
	\item Large-scale deployment;
	
	\item Identifier outlier sources;
	
\end{itemize}

\begin{framed}
	conclusion Copy paste from zhang:
	
	Thus, the main challenge faced by outlier detection techniques for WSNs is to satisfy the mining accuracy requirements while maintaining the resource consumption of WSNs
	to a minimum [21]. In other words, the main question is how to
	process as much data as possible in a decentralized and online
	fashion while keeping the communication overhead, memory
	and computational cost low [1].
\end{framed}


\section{Classification of outlier}

Zhang et al. <zhang2010> presents aspects to be used as metrics to compare characteristics of different outlier detection techniques. In parallel, Chandola et al. <chandola2009> presents a similar approach for the classification of outlier detection. In table \ref{table:t1} is present a comparison between two approaches to classify the nature of input sensor data.

	
% Please add the following required packages to your document preamble:
% \usepackage{multirow}
\begin{table}[]
	\centering
	\caption{My caption}
	\label{my-label}

	\begin{tabular}{ llllll }
		\hline
		\multicolumn{3}{| c |}{Zhang et al.}                                                                                                                                                                          & \multicolumn{3}{ c |}{Chandola et al.}                                                                                                                                                                                                                                                                                                                                               \\ \hline
		\multicolumn{1}{|l}{\multirow{8}{*}{Input sensor data}} & \multirow{2}{*}{attributes}   & \multicolumn{1}{l|}{\multirow{2}{*}{univariate/multivariate}}                                                     & \multirow{8}{*}{Nature of input data} & \multirow{2}{*}{Described using attributes} & \multicolumn{1}{l|}{different types (binary, categorical, continuous)}                                                                                                                                                                                                                       \\ \cline{6-6} 
		\multicolumn{1}{|l}{}                                   &                               & \multicolumn{1}{l|}{}                                                                                             &                                       &                                             & \multicolumn{1}{l|}{quantity: i) univariate; ii) multivariate w/ same type; iii) multivariate w/ different data types;}                                                                                                                                                                      \\ \cline{2-3} \cline{5-6} 
		\multicolumn{1}{|l}{}                                   & \multirow{6}{*}{correlations} & \multicolumn{1}{l|}{dependencies among the attribures sensor nodes}                                               &                                       & \multirow{3}{*}{Related to each other}      & \multicolumn{1}{l|}{In sequence data, the data instances are linearlyordered, for example, time-series data, genome sequences, and protein sequences.}                                                                                                                                       \\ \cline{3-3} \cline{6-6} 
		\multicolumn{1}{|l}{}                                   &                               & \multicolumn{1}{l|}{\multirow{5}{*}{dependency of sensor node readings on history and neighboring node readings}} &                                       &                                             & \multicolumn{1}{l|}{In spatial data, each data instance is related to its neighboring instances, for example,vehicular traffic data, and ecological data. When the spatial data has a temporal (sequential) component it is referred to as spatio-temporal data, for example, climate data.} \\ \cline{6-6} 
		\multicolumn{1}{|l}{}                                   &                               & \multicolumn{1}{l|}{}                                                                                             &                                       &                                             & \multicolumn{1}{l|}{In graph data, data instances are represented as vertices in a graph and areconnected to other vertices with edges. Later in this section we will discuss situations where such relationships among data instances become relevant for anomalydetection.}                \\ \cline{5-6} 
		\multicolumn{1}{|l}{}                                   &                               & \multicolumn{1}{l|}{}                                                                                             &                                       & Relationship                                & \multicolumn{1}{l|}{Can be categorized based on relationship present among data instances}                                                                                                                                                                                                   \\ \cline{5-6} 
		\multicolumn{1}{|l}{}                                   &                               & \multicolumn{1}{l|}{}                                                                                             &                                       & \multirow{2}{*}{Applicability}              & \multicolumn{1}{l|}{for statistical techniques}                                                                                                                                                                                                                                              \\ \cline{6-6} 
		\multicolumn{1}{|l}{}                                   &                               & \multicolumn{1}{l|}{}                                                                                             &                                       &                                             & \multicolumn{1}{l|}{for nearest-neighbor-based techniques}                                                                                                                                                                                                                                   \\ \hline
		&                               &                                                                                                                   &                                       &                                             &                                                                                                                                                                                                                                                                                             
	\end{tabular}
\end{table}
	

Based on the work of Zhang et al. and Chandola et al., the table \ref{table:t2} identifies the different types of outliers. 
Those types differs on the objective of the outlier detection techniques: detect anomalies in individual data instances or in groups of data to detect irregularities, respectively, in local or in the global measuring system.

% Please add the following required packages to your document preamble:
% \usepackage{multirow}
\begin{table}[h!]
	\centering
	\caption{Classification of the outlier techniques based on the type of the outlier/anomaly.}
	\label{table:t2}


\resizebox{1.0\textwidth}{!}{ 

	\begin{tabular}{c|c|c|c|c|c}
		\hline
		\multicolumn{3}{c|}{Zhang et al.}                                                                                                                                                                                                                                                                                                                                        & \multicolumn{3}{c|}{Chandola et al.}                                                                                                                                                                                                                                                                                                                                   \\ \hline
		\multicolumn{1}{|c|}{\multirow{6}{*}{\begin{tabular}[c]{@{}c@{}}Type\\ of\\ outliers\end{tabular}}} & \multirow{2}{*}{\begin{tabular}[c]{@{}c@{}}Local\\ outliers\end{tabular}}  & \begin{tabular}[c]{@{}c@{}}Variation 1:\\ anomalous values detection\\ only depends on its historical values\end{tabular}                                                             & \multirow{6}{*}{\begin{tabular}[c]{@{}c@{}}Type\\ of\\ anomaly\end{tabular}} & \multirow{2}{*}{\begin{tabular}[c]{@{}c@{}}Point\\ anomalies\end{tabular}}      & \multicolumn{1}{c|}{\multirow{2}{*}{\begin{tabular}[c]{@{}c@{}}An individual data instance\\ is considered anomalous,\\ with respect to the others\end{tabular}}}                                     \\ \cline{3-3}
		\multicolumn{1}{|c|}{}                                                                              &                                                                            & \begin{tabular}[c]{@{}c@{}}Variation 2;\\ anomalous values detection\\ depends on historical values\\ and on values of neighboring\end{tabular}                                       &                                                                              &                                                                                 & \multicolumn{1}{c|}{}                                                                                                                                                                                 \\ \cline{2-3} \cline{5-6} 
		\multicolumn{1}{|c|}{}                                                                              & \multirow{3}{*}{\begin{tabular}[c]{@{}c@{}}Global\\ outliers\end{tabular}} & \begin{tabular}[c]{@{}c@{}}Variation 1:\\ All data is transmitted\\ to a centralized architecture\\ where outlier detection \\ techniques takes place\end{tabular}                    &                                                                              & \multirow{3}{*}{\begin{tabular}[c]{@{}c@{}}Contextual\\ anomalies\end{tabular}} & \multicolumn{1}{c|}{\begin{tabular}[c]{@{}c@{}}Contextual attributes:\\ are used to determine \\ the context for a given instance\end{tabular}}                                                       \\ \cline{3-3} \cline{6-6} 
		\multicolumn{1}{|c|}{}                                                                              &                                                                            & \begin{tabular}[c]{@{}c@{}}Variation 2:\\ Data from a cluster of sensors\\ is used for outlier detection\\ in a aggregate/clustering\\  based architecture\end{tabular}               &                                                                              &                                                                                 & \multicolumn{1}{c|}{\multirow{2}{*}{\begin{tabular}[c]{@{}c@{}}Behavioral attributes:\\ defines the noncontextual \\ characteristics of a \\ given instance.\end{tabular}}}                           \\ \cline{3-3}
		\multicolumn{1}{|c|}{}                                                                              &                                                                            & \begin{tabular}[c]{@{}c@{}}Variation 3:\\ Individual nodes can identify\\ global outliers if they have\\  a copy of global estimator model\\ obtained from the sink node\end{tabular} &                                                                              &                                                                                 & \multicolumn{1}{c|}{}                                                                                                                                                                                 \\ \cline{2-3} \cline{5-6} 
		\multicolumn{1}{|c|}{}                                                                              & \multicolumn{2}{c|}{}                                                                                                                                                                                                                                              &                                                                              & \begin{tabular}[c]{@{}c@{}}Collective\\ anomalies\end{tabular}                  & \multicolumn{1}{c|}{\begin{tabular}[c]{@{}c@{}}If a collection of related data\\ instances is anomalous\\ with respect to the entire data set,\\ it is defined as a collective anomaly.\end{tabular}} \\ \hline
	\end{tabular}

}
\end{table}

\newpage


Table \ref{table:t3} continues the classification, focusing in three parts: 
\begin{itemize}
		\setlength\itemsep{-0.5em}
		\item The need of pre-classified data (to implement supervised, semi-supervised or unsupervised outlier detection techniques);
		
		\item The output of outlier detection techniques (binary labels for normal/abnormal data-set and a score for each data-set to evaluate the weight of being an anomaly)
		
		\item The identity of the outliers (detect errors, events or malicious attacks)
\end{itemize}

% Please add the following required packages to your document preamble:
% \usepackage{multirow}
\begin{table}[h!]
\centering
\caption{ Classification of outlier detection techniques according to: i) need of pre-classified data; ii) output of detection techniques; iii) identity of outliers}
\label{table:t3}


\resizebox{1.0\textwidth}{!}{ 
	
\begin{tabular}{c|c|c|c|c|c}
	\hline
	\multicolumn{3}{c|}{Zhang et al.}                                                                                                                                                                                                                                                                                     & \multicolumn{3}{c|}{Chandola et al.}                                                                                                                                                                                                                                                                                                                                             \\ \hline
	\multicolumn{1}{|c|}{\multirow{3}{*}{\begin{tabular}[c]{@{}c@{}}Availability\\ of\\ pre-defined\\ data\end{tabular}}} & Supervised                                                  & \begin{tabular}[c]{@{}c@{}}Require pre-classified\\ normal and abnormal data\end{tabular}                                       & \multirow{3}{*}{\begin{tabular}[c]{@{}c@{}}Data\\ labels:\\ normal or\\ anomalous\end{tabular}} & \begin{tabular}[c]{@{}c@{}}Labels obtained by\\ Supervised\\ Anomaly Detection\end{tabular}       & \multicolumn{1}{c|}{\begin{tabular}[c]{@{}c@{}}Training data has labeled instances \\ for normal and anomalous classes\end{tabular}}                                       \\ \cline{2-3} \cline{5-6} 
	\multicolumn{1}{|c|}{}                                                                                                & Semi-supervised                                             & \begin{tabular}[c]{@{}c@{}}Require only pre-classified\\ normal data\end{tabular}                                               &                                                                                                 & \begin{tabular}[c]{@{}c@{}}Labels obtained by\\ Semi-supervised \\ Anomaly Detection\end{tabular} & \multicolumn{1}{c|}{\begin{tabular}[c]{@{}c@{}}Training data has labeled instances\\ only for normal class.\\ There is no labels for the\\ anomalous classes\end{tabular}} \\ \cline{2-3} \cline{5-6} 
	\multicolumn{1}{|c|}{}                                                                                                & Unsupervised                                                & Do not require pre-classified data                                                                                              &                                                                                                 & \begin{tabular}[c]{@{}c@{}}Labels obtained by\\ Unsupervised\\ Anomaly Detection\end{tabular}     & \multicolumn{1}{c|}{\begin{tabular}[c]{@{}c@{}}Techniques that do not require\\ training data\end{tabular}}                                                                \\ \hline
	\multicolumn{1}{|c|}{\multirow{2}{*}{\begin{tabular}[c]{@{}c@{}}Degree of\\ being an \\ outlier\end{tabular}}}        & Scalar                                                      & \begin{tabular}[c]{@{}c@{}}Zero-one classification:\\ Classifies a data measurement\\ into normal or outlier class\end{tabular} & \multirow{5}{*}{\begin{tabular}[c]{@{}c@{}}Output\\ of\\ Anomaly\\ detection\end{tabular}}      & \multirow{2}{*}{Scores}                                                                           & \multicolumn{1}{c|}{\multirow{2}{*}{\begin{tabular}[c]{@{}c@{}}Degree of which a data instance\\ is consider an anomaly\end{tabular}}}                                     \\ \cline{2-3}
	\multicolumn{1}{|c|}{}                                                                                                & Score                                                       & \begin{tabular}[c]{@{}c@{}}Assign to each data measurements\\ a outlier score;\\ Display a ranked list of outliers\end{tabular} &                                                                                                 &                                                                                                   & \multicolumn{1}{c|}{}                                                                                                                                                      \\ \cline{1-3} \cline{5-6} 
	\multicolumn{1}{|c|}{\multirow{3}{*}{\begin{tabular}[c]{@{}c@{}}Identity\\ of\\ outliers\end{tabular}}}               & Errors                                                      & \begin{tabular}[c]{@{}c@{}}Noise-related measurement\\ or data coming from a faulty sensor\end{tabular}                         &                                                                                                 & \multirow{3}{*}{Labels}                                                                           & \multicolumn{1}{c|}{\multirow{3}{*}{\begin{tabular}[c]{@{}c@{}}Provide binary labels\\ (normal/anomalous)\end{tabular}}}                                                   \\ \cline{2-3}
	\multicolumn{1}{|c|}{}                                                                                                & Events                                                      & \begin{tabular}[c]{@{}c@{}}Particular phenomena\\ that changes the real-world state\end{tabular}                                &                                                                                                 &                                                                                                   & \multicolumn{1}{c|}{}                                                                                                                                                      \\ \cline{2-3}
	\multicolumn{1}{|c|}{}                                                                                                & \begin{tabular}[c]{@{}c@{}}Malicious\\ attacks\end{tabular} & \begin{tabular}[c]{@{}c@{}}Related to network security issues \\ (Outside of the scope of the paper)\end{tabular}                               &                                                                                                 &                                                                                                   & \multicolumn{1}{c|}{}                                                                                                                                                      \\ \hline
\end{tabular}

}

\end{table}

\newpage

\section{Taxonomy of Outlier Detection Techniques}

The study of detection techniques requires a well defined taxonomy framework that addresses the related work on the different areas. This taxonomy is well defined and solid in the literature, where the works of Zhang et al. and Chandola et al. reflect a similar approach on presenting a taxonomy for outlier detection techniques.

In the following sections a coverage in relevant techniques is presented:

\begin{itemize}
	\setlength\itemsep{-0.5em}
	\item Classification based techniques.
	\subitem Bayesian Networks
	\subitem Rule-based techniques
	\subitem Support Vector Machines
	
	\item Statistical based techniques.
	\subitem Parametric --- Gaussian based
	\subitem Non-parametric --- Histogram based
	\subitem Non-parametric --- Kernel function based
	
	\item Nearest Neighbor-based techniques.
	\subitem Using distance
	\subitem Using relative density
	
	\item Clustering based techniques.
	
	\item Spectral Decomposition based techniques.
	
\end{itemize}


%%%o resto deste capítulo está dividido em ficheiros










