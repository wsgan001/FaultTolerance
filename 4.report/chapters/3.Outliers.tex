%\lipsum[4-4]
In this chapter it is  made the study of the state of the art of outliers and it's relevance in railways.


\section{Definition of outlier detection}

Outlier detection is a computational task to detect and retrieve information from erroneous data values. The definition is usually close to anomaly detection or deviation detection. 


%%%%%%%%%%%%%%%%%%%%%
%%   Wireless sensor networks
%%%%%%%%%%%%%%%%%%%%%
\section{Outlier detection in WSNs}

Wireless sensor networks (WSNs) has been widely used in several applications in several domains such as industrial, scientific, medical and others. Those applications has been supported by the advances in wireless technologies as well as in the evolution of microcontroller technologies, with enhanced processing capabilities associated with reduced energy consumption.

\subsection{Motivation}

"In sensor networks, the majority of the energy is consumed in radio communication rather than computation" ... in the particular case of Sensoria sensors and Berkeley motes, the ratio of energy consumption between computation and communication modes is between 1000 and 10000 <rajasegarar2007>. Thus, an research opportunity is raised to reduce the communication usage of $\mu C$s by adding processing features towards the redution of energy consumption.

The motivation of detecting outliers in data acquired from WSNs has been extensivelly presented in the literature. The need for acquire data from harsh or "highly dynamic" environments as well as the need to validate and extract knowledge from the acquired data are one of the main points in the motivation to study the outlier detection in WSNs. <zang2010> <chandola2009> <ghorbel2015> <martins2015>  

\subsection{Research areas}
Zhang et al. <zhang2010> identifies the outlier detection research areas in three domains: 

\begin{itemize}
	\item Intrusion detection: Situation caused by malicious attacks, where the detection techniques are query-driven techniques;
	
	\item Fault detection: Situation where the data suffer from noise and errors and where the detection techniques are data-driven ones;
	
	\item Event detection: Situation caused by the occurrence of one atomic or multiple events and where the majority of the research has been developed due to the complexity of detecting and extracting information on ?? upper layers ??
\end{itemize}

Based on the division of this three domains, the upcoming research is intended to be focused on the event detection techniques. ?? The railway environment requires closed subsystems that meets specific standards. Despite the intrusion detection should be considered, this must be took into consideration accordingly to the development and implementation of the wireless smart metering system for the railways. ?? The fault detection should and must be taken into consideration and the data outcome must be, preferably, a null value with a warning raised. ???

\subsection{Challenges}

The challenges of outlier detection in WSNs are related to the quality of the acquisition of the sensors, the fiability of the modules in terms of energy or environmental susceptibility, and the communication requirements and restrictions.

Zhang et al. <zhang2010> lists the challenges as the following:

\begin{itemize}
	\setlength\itemsep{-0.5em}
	
	\item Resource constraints;
	
	\item High communication costs;
	
	\item Distributed streaming data;
	
	\item Dynamic network topology, \\ frequent communication failures, \\ mobility and heterogeneity of nodes;
	
	\item Large-scale deployment;
	
	\item Identifier outlier sources;
	
\end{itemize}

\begin{framed}
	Thus, the main challenge faced by outlier detection techniques for WSNs is to satisfy the mining accuracy requirements while maintaining the resource consumption of WSNs
	to a minimum [21]. In other words, the main question is how to
	process as much data as possible in a decentralized and online
	fashion while keeping the communication overhead, memory
	and computational cost low [1].
\end{framed}


\section{Classification of outlier}

\section{Taxonomies}
