

This work presents the study of outlier detection techniques, covering the state of the art and focusing the study towards the implementation of those techniques in a railways WSN.

In the perspective of fault-tolerance in computing systems domain, those techniques are of extreme interest to avoid the unwanted failures in computing systems.
Following the taxonomy extensively presented in the literature, an outlier is considered as a fault in the input of the system. This fault will be a cause of an error and, without a outlier detection mechanism, those errors can be propagated to the outer frontier of the system, resulting on failure, or in a better definition, resulting in a behavior that is not according to the specifications. 
The task of those techniques is to detect the outliers in the computing system and avoid them to be propagated to the output of the system. 

Based on this domain of fault-tolerance in computing systems, the definition of frontier of the railways WSN was proposed. In particular, the railways WSN as previously presented, provides data and information for a Decision Support System (DSS).
Rather than considering only the data provided from the sensor network, an important step is to evaluate and avoid failures in the entire system (constituted by the sensing subsystem - that collects the data from the environment - and the DSS subsystem - that generates decision information based on the available data). 

As a starting point for future research, with this work two research questions was presented towards deepen this domain. An iterative procedure is expected to be taken by continuously searching in the literature for new improvements on this domain and continuous to deepen towards an new contribution. At this moment, a methodology is presented for the immediate future to implement an outlier detection technique in a energy monitoring test-bench.