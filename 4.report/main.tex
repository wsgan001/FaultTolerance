\documentclass{report}%{IEEEtran}
\usepackage[utf8]{inputenc}
\usepackage[square]{natbib}
\usepackage{graphicx}
\usepackage{framed}
\usepackage{multirow}
\usepackage{lipsum}  
\usepackage{verbatim}

\usepackage[oneside,width=17.5cm,height=24cm,left=2cm]{geometry}
%\usepackage[nolist,nohyperlinks]{acronym}

\usepackage[]{nomencl}

%\usepackage{biblatex}
%\addbibresource{references.bib}

\usepackage{array}
\newcolumntype{C}[1]{>{\raggedright\let\newline\\\arraybackslash\hspace{0pt}}m{#1}}


\title{\vspace{-3.5cm} Fault Tolerance in Computational Systems - Report \\
\hrulefill\\
 Influence of outliers in a railway remote monitoring system%\\
%Giuseppe Lipari, Giorgio Buttazzo, Luca Abeni.
}
\author{Student: Vítor A. Morais\\
Supervisor: António Pina Martins }



\makenomenclature
%\newacronym{svm}{svm}{support vector machine}

%\newacronym{wsn}{WSN}{Wireless Sensor Network}

\nomenclature{\textbf{WSNs}}{Wireless Sensor Networks}


\begin{document}

\maketitle



%\chapter*{Abstract}
%Abstract goes here
%%
%%   Section 
%%
\tableofcontents


\printnomenclature

\chapter*{Symbols}
\input{chapters/B.symbols}

\chapter{Introduction}
%\lipsum[4-4]
This chapter presents the context, motivation and document structure of a study on smart metering and communication protocols used in smart grids. 

\section{Context and motivation}

Smart grids are conceived as electric grids that deliver electricity from generation points to consumers, having the feature of controlling the entire process.

In railways...

Outliers are bla bla,.,.

The study of outliers is relevant due to it's influence in ....

With this work it is expected to raise the awareness of outliers detection in the phd study




\section{Document structure}

This document is divided in 4 chapters, each of them incorporate the relevant subsections to present the subjects mentioned
%contains several subsections according to the subjects mentioned.

\begin{table}[!h]
    \label{tb:struct}
    \centering
    \caption{Document structure}
    \vspace{0.2em}
    \begin{tabular}{c|l}%{C{2cm}|C{9cm}}
    \textbf{Chapter} & \textbf{Title}                    \\ \hline
    1       &                   Introduction             \\ \hline
    2       &                   Railways Remote Monitoring Systems       \\ \hline
    3       &                   Outliers Detection    \\ \hline
    4       &                   Conclusions               \\
    \end{tabular}
\end{table}


\chapter{Railways Remote Monitoring Systems}
\input{chapters/2.Railways}

\chapter{Outliers Detection}
%\lipsum[4-4]
In this chapter it is  made the study of the state of the art of outliers and it's relevance in railways.


\section{Definition of outlier detection}

Outlier detection is a computational task to detect and retrive information from erroneous data values. The definition is usually close to anomaly detection or deviation detection. 


%%%%%%%%%%%%%%%%%%%%%
%%   Wireless sensor networks
%%%%%%%%%%%%%%%%%%%%%
\section{Outlier detection in WSNs}

Wireless sensor networks (WSNs) has been widely used in several applications in several domains such as industrial, scientific, medical and others. Those applications has been supported by the advances in wireless technologies as well as in the evolution of microcontroller technologies, with enhanced processing capabilities associated with reduced energy consumption.

\subsection{Motivation}

"In sensor networks, the majority of the energy is consumed in radio communication rather than computation" ... in the particular case of Sensoria sensors and Berkeley motes, the ratio of energy consumption between computation and communication modes is between 1000 and 10000 <rajasegarar2007>. Thus, an research opportunity is raised to reduce the communication usage of $\mu C$s by adding processing features towards the redution of energy consumption.

The motivation of detecting outliers in data acquired from WSNs has been extensivelly presented in the literature. The need for acquire data from harsh or "highly dynamic" environments as well as the need to validate and extract knowledge from the acquired data are one of the main points in the motivation to study the outlier detection in WSNs. <zang2010> <chandola2009> <ghorbel2015> <martins2015>  

\subsection{Research areas}
Zhang et al. <zang2010> identifies the outlier detection research areas in three domains: 

\begin{itemize}
	\item Intrusion detection: Situation caused by malicious attacks, where the detection techniques are query-driven techniques;
	
	\item Fault detection: Situation where the data suffer from noise and errors and where the detection techniques are data-driven ones;
	
	\item Event detection: Situation caused by the occurrence of one atomic or multiple events and where the majority of the research has been developed due to the complexity of detecting and extracting information on xc 
\end{itemize}}


\subsection{Challenges}



\chapter{Future Research}
%\lipsum[4-4]
In this chapter there are presented the future steps in research on outliers detection on railways WSN-based smart grid.


%%%%%%%%%%%%%%%%%%%%%
%%   Smart metering Systems
%%%%%%%%%%%%%%%%%%%%%
\section{Outlier techniques selected}

\lipsum[1]


\section{Synthesis}

\lipsum[1]


\chapter{Conclusion}
The 21st century brought several challenges to improve energy consumptions, increase the usage of renewable energy sources, increase the quality of service, reduce the operation costs, etc. All of these challenges are proposed to reduce the fossil energy dependence, reduce the energy consumption and minimize the environment damaging.

The European Commission 2020 targets are clear: reducing 20\% of greenhouse emissions, increase 20\% of energy from renewable sources and improve 20\% of energy efficiency. Smart Grids are being developed to be a technology that, by increasing the energy efficiency and increasing the usage of renewable energy sources, it is possible to reduce the greenhouse gas emissions and fulfill the 2020 targets.

An important part of the SG are the metering systems. Starting with smart (and non-smart) meters, the electrical energy measurement devices was presented, covering the background behind smart meters and finalizing with regulations and directives concerning these devices.
The Smart Metering Infrastructure is presented to have a higher view of integration in the Smart Grid. Followed is the Energy Management Systems and Energy Information Systems. A cost-analysis is necessary to evaluate the Smart Metering Systems integration in the traditional electrical grid. 

On the field of the communications, a three stage  state of the art covering approach was followed: requirements, followed by the presentation of the communication technologies and ending with the standards. The content of this paper has ended with the presentation of some opportunities and trends on research of smart metering, giving a focus to present the opportunities in the railway environments.
To conclude, the future of smart metering systems is bright with the incentives from the European Commission, specifically from Shift2Rail program.


\bibliographystyle{IEEEtran}
\bibliography{IEEEabrv,references}

%\bibliographystyle{plainnat}
%\bibliography{references}


\end{document}